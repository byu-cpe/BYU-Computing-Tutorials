\input{../../../Academic/latex/asst_template.tex}


\class{ECEN 625}
\term{Winter 2019}
\assttitle{Vivado + Vivado HLS Tutorial}
%\duedate{Tuesday, February 26, 2019 11:59pm}

\newlength\myheight
\newlength\mydepth
\settototalheight\myheight{Xygp}
\settodepth\mydepth{Xygp}
\setlength\fboxsep{0pt}
\newcommand*\inlinegraphics[1]{%
  \settototalheight\myheight{Xygp}%
  \settodepth\mydepth{Xygp}%
  \raisebox{-\mydepth}{\includegraphics[height=\myheight]{#1}}%
}

\setuppage

\makeatletter
\@ifpackageloaded{biblatex}{\addbibresource{references.bib}}{\bibliography{references}}
\makeatother

\begin{document}

\maketitle
\thispagestyle{fancy}

\section{Setup}
You can install Vivado on your local machine (\url{https://www.xilinx.com/support/download.html}).  If you do this, I suggest you install Vivado 2018.2 Design or System Edition on an Ubuntu 16.04 machine.  If you prefer, you can install Vivado on a Windows machine.  I haven't tested this.  

Vivado HLS requires a license.  You can access the department Xilinx license server by setting the following environment variable.  This means you must either be on the university network, or connected to the CAEDM VPN.

\begin{lstlisting}
export LM_LICENSE_FILE=2100@ece-xilinx.byu.edu
\end{lstlisting}

\subsection{Installing Boards}
If you are using a Digilent board, such as the Zedboard, you need to setup the board files in Vivado.  See \url{https://github.com/Digilent/vivado-boards/}.

\subsection{Running Vivado}
Before you can run the Vivado tools you should first run the configuration script:
\begin{lstlisting}
source /opt/Xilinx/Vivado/2018.2/settings64.sh
\end{lstlisting}

This will add the tools to your {\tt PATH}.  If you don't want to setup your own machine, contact me and I can give you access to a server with Vivado 2018.2 installed.

To run Vivado, simply type
\begin{lstlisting}
vivado
\end{lstlisting}


\section{Hello World (Hardware)}
\subsection{Creating the Project}
After launching Vivado, follow these steps to create a hardware project:
\begin{enumerate}
\item Create new project..., and choose a project name and location.  Next.  Choose an RTL project. Next.  
\item On the next screen, click \emph{Boards} at the top, and choose your board (ie. Zedboard).  Click Finish.
\end{enumerate}

\subsection{Creating a Base Design}
In these steps we will create a basic system, containing only the Zynq processing system (PS).
\begin{enumerate}
	\item Click \emph{Create Block Design}, and click \emph{OK} on the popup.
	\item Add the \emph{ZYNQ7 Processing System} IP to the design (right-click, Add IP).
	\item A green banner should appear with a link to \emph{Run Block Automation}.  Run this. This will configure the ZYNQ for your board.
	\item The \emph{FCLK\_CLK0} output of the PS will serve as your system clock.  It is set to 100MHz by default.  Connect it to the \emph{M\_AXI\_GP0\_ACLK} input.	
	\item Generate a top-level module: In the \emph{Sources} window, right-click on your block design (\emph{design\_1.bd}) and select \emph{Create HDL Wrapper}. Use the option to \emph{Let Vivado manager wrapper and auto-update}.
\end{enumerate}

\subsection{Synthesizing the hardware}
\begin{enumerate}
	\item Run \emph{Generate Bitstream}.
	\item Once the bitstream generation is complete, export the hardware.  File$\rightarrow$Export$\rightarrow$Export Hardware.  Check \emph{Include Bitstream}, and choose a location to store the Hardware Description File (.hdf). 
\end{enumerate}


\section{Hello World (Software)}

Run the Xilinx SDK (\texttt{xsdk}), and choose a workspace location.

\subsection{Create SDK Projects}
\begin{enumerate}
	\item Create the hardware project.  New Project$\rightarrow$Xilinx$\rightarrow$Hardware Platform Specification.  Chose a project name, and indicate the location of your exported .hdf file.
	\item Create the board support package project.  This will includes all of the system support software (drivers, etc.).  New Project$\rightarrow$Xilinx$\rightarrow$Board Support Package.  Indicate the hardware platform project you created in the last step, and choose the \emph{standalone} OS. 
	\item In the \emph{Board Support Package Settings} popup, go to the \emph{standalone} menu, and change \emph{stdout} to use \emph{coresight\_comp\_0}.  This ensures that when your application prints to stdout, it will be sent over the virtual console over the JTAG, and not on the physical UART pins present on the board.  
	\item Create your application project.  New Project$\rightarrow$Xilinx$\rightarrow$Applicaton Project.  Choose an application name (ie. HelloWorld), and configure the project to use the Board Support Package project you created in the last step.
	\item Open and inspect the code found in \texttt{src/helloworld.c}.
\end{enumerate}

\subsection{Run the Applicaton on the Board}
\begin{enumerate}
	\item Configure the FPGA with the \inlinegraphics{configure_fpga.png} menu button.
	\item Right-click on your application project, choose Run As$\rightarrow$Launch on Hardware (System Debugger).
	\item To view the program output, you will need to use the console selector button \inlinegraphics{console_select.png} in the \emph{Console} window to select the \emph{TCF Debug Virtual Terminal - ARM Cortex-A9 MPCore \#0} console.
	\item You should see the message \emph{Hello World}.
\end{enumerate}


\section{Exporting your HLS as IP}
This section discusses how you can export your IP from Vivado HLS to be used in a Vivado project.  Since our goal is to communicate with the HLS IP from software, we will add a Slave AXI connection to our HLS IP core so that it can be connected to the ARM AXI bus.

\begin{enumerate}
	\item Run Vivado HLS and open your project from the last assignment.
	\item In Vivado HLS, add a directive to your top-level hardware function.  Choose the \emph{INTERFACE} directive type, and change the mode to an AXI4-Lite Slave (\emph{s\_axilite}).
	\item Run C Synthesis.
	\item Click the Export RTL button \inlinegraphics{export_rtl.png}, and make sure the Format Selection is set to \emph{IP Catalog}.
	\item Close Vivado HLS.
\end{enumerate}

\section{Adding your IP to Vivado}
\begin{enumerate}
	\item Launch Vivado, open your existing project, and open the block design.
	\item If you do not already have a \emph{Processor System Reset} IP, add one to your design.  This will use the reset signal output by the processing system to reset IP in the FPGA fabric.  
	
	\begin{enumerate}
		\item Connect the system clock (\emph{FCLK\_CLK0} from PS) to the \emph{slowest\_sync\_clk} input.
		\item Connect the processor reset output (\emph{FCLK\_RESET0\_N}) to the \emph{ext\_reset\_in} input.		
	\end{enumerate}
	\item If you do not already have a \emph{AXI Interconnect} IP, add one to your design.  This is the bus that will allow the ARM CPU to communicate with the IP implemented in the FPGA fabric.
	
	\begin{enumerate}
		\item Configure the bus to have 1 Slave Interface and 1 Master Interface.
		\item Connect the PS bus master (\emph{M\_AXI\_GP0} from PS) to the \emph{S00\_AXI} slave port.
		\item Connect your clock (\emph{FCLK\_CLK0} from PS) to all the clock inputs (\emph{*ACLK})
		\item Connect your interconnect reset (\emph{interconnect\_aresetn} from \emph{Processor System Reset}) to the \emph{ARESETN} input.
		\item Connect your peripheral reset (\emph{peripheral\_aresetn} from \emph{Processor System Reset}) to the other reset inputs (\emph{*\_ARESETN})
	\end{enumerate}


	\item {Add your HLS IP:}
	
	\begin{enumerate}
		\item Open the IP catalog
		\item Right-click, \emph{Add Repository}
		\item Navigate to your HLS IP found in \texttt{your\_hls\_project/your\_solution/impl/ip}.
		\item Go back to your block design and add the HLS IP to your design.
	\end{enumerate}
	
	\item{Connect up your HLS IP:}
	
	\begin{enumerate}
		\item  Connect the clock (\emph{FCLK\_CLK0} from PS) to the clock input (\emph{ap\_clk})
		\item  Connect the reset (\emph{peripheral\_aresetn} from \emph{Processor System Reset}) to the reset input (\emph{ap\_rst\_n})
		\item Connect the bus (\emph{M00\_AXI} from the \emph{AXI Interconnect}) to the bus slave port (\emph{s\_axi\_AXILiteS})
	\end{enumerate}
	
	\item Assign an address to your HLS IP.  Open the \emph{Address Editor}, find your IP, right-click \emph{Assign Addresses}.
	
	\item Run \emph{Generate Bitstream.}
	
	\item Export the hardware with bitstream, and be sure to choose the same location as last time, overwritting the existing hardware description file.

	\item Close Vivado
\end{enumerate}


\section{Communicating with your HLS IP from Software}
\begin{enumerate}
	\item Launch Xilinx SDK and reopen your existing workspace.
	\item You should see a prompt indicating that the hardware has changed.  Click \emph{Yes} to update your software libraries to support the new hardware system.
	\item Right-click on your BSP project and click \emph{Re-generate BSP sources}. 
	\item Right-click on your BSP project, click \emph{Board Support Package Settings}, go to drivers and make sure you can find your IP core listed.  For the in-class demo, my IP core was called \emph{sum\_array}, so I can find the IP named \emph{sum\_array\_0} in the list and verify that the driver called \emph{sum\_array} is selected. Click OK to close the settings window.
	\item In your BSP project, you can look in \emph{ps7\_coretexa9\_0/include} to find the header files for your driver.  Mine are called \emph{xsum\_array.h} and \emph{xsum\_array\_hw.h}.
	\item Include the necessary header file in your application code and write software to test that you can start your IP, wait for it to complete, and retrieve the return value.  As with most Xilinx drivers, you will need to call the initialization function first (\emph{XSum\_array\_Initialize}).  The first argument is a struct variable that you should define and pass in by pointer, the second is the device ID (\emph{ XPAR\_XSUM\_ARRAY\_0\_DEVICE\_ID}), that can be used by including \emph{xparameters.h}.
\end{enumerate}


\renewcommand*{\bibfont}{\footnotesize}
\printbibliography[heading=bibintoc]

\end{document}

